%%%%%%%%%%%%%%%%%%%%%%%%%%%%%%%%%%%%%%%%%
% Medium Length Professional CV
% LaTeX Template
% Version 2.0 (8/5/13)
%
% This template has been downloaded from:
% http://www.LaTeXTemplates.com
%
% Original author:
% Trey Hunner (http://www.treyhunner.com/)
%
% Important note:
% This template requires the resume.cls file to be in the same directory as the
% .tex file. The resume.cls file provides the resume style used for structuring the
% document.
%
%%%%%%%%%%%%%%%%%%%%%%%%%%%%%%%%%%%%%%%%%

%----------------------------------------------------------------------------------------
%	PACKAGES AND OTHER DOCUMENT CONFIGURATIONS 
%----------------------------------------------------------------------------------------
 
\documentclass{resume} % Use the custom resume.cls style 
\usepackage[left=0.6in, top=0.65in, right=0.6in, bottom=0.65in]{geometry} % Document margins
\usepackage{helvet}
\usepackage{fancyhdr}
\pagestyle{fancy}
\usepackage{lastpage}
\fancyhf{}
\renewcommand{\headrulewidth}{0pt}
\fancyhead{}
\rfoot{\scriptsize \textit{Page \thepage of \pageref{LastPage}. Last updated: May 24, 2019}}

\renewcommand{\familydefault}{\sfdefault}
\newenvironment{myitemize}
{ \begin{itemize}[leftmargin=0.2em,label={}]
    \setlength{\itemsep}{0pt}
    \setlength{\parskip}{0pt}
    \setlength{\parsep}{0pt}     }
{ \end{itemize}                  } 


\usepackage{url}
\usepackage{hyperref}
\usepackage{enumitem,kantlipsum}
\newcommand{\tab}[1]{\hspace{.2667\textwidth}\rlap{#1}}
\newcommand{\itab}[1]{\hspace{0em}\rlap{#1}}
\name{Jianfeng Chen} % Your name 
% \address{\url{http://jianfeng.us}}
% \address{3111-A Walnut Creek Parkway, Raleigh, NC, 27606}
\address{(919)-457-2034 \\ \url{https://jianfeng.us}   \\ \href{mailto:jchen37@ncsu.edu}{jchen37@ncsu.edu}~$\bullet$~\href{mailto:ginfungchan@gmail.com}{ginfungchan@gmail.com}
}

\begin{document}
%----------------------------------------------------------------------------------------
%	OBJECTIVE
%----------------------------------------------------------------------------------------
% \begin{rSection}{Objectives}
% \textit{\textbf{Objective: Seeking SDE or Machine Learning internship in Summer'18 and full-time positions starting May'19.
% }}
% \end{rSection}

%----------------------------------------------------------------------------------------
%	EDUCATION SECTION
%----------------------------------------------------------------------------------------
\vspace{2em}

\begin{rSection}{Education}

{\bf PhD in Computer Science} \hfill {Aug 2014 - May 2019}\\ 
North Carolina State University, GPA: 4.0/4.0  \\
Dissertation: On the Value of Sampling and Pruning for Search-Based Software Engineering

{\bf Master in Computer Science {\it (En route)}} \hfill {Aug 2014 - Dec 2018}\\ 
North Carolina State University, GPA: 4.0/4.0  \\
Coursework: Data-to-Knowledge $|$ DevOps $|$ Advanced AI $|$ Algorithm Analysis $|$ Data Mining $|$ Automated SE

{\bf BS in Computer Science} \hfill {Sep 2010 - May 2014}\\ 
Shandong University, China, GPA: 91.1/100  \\
Coursework: Data Structure $|$ OS $|$ Networking $|$ Database System $|$ Numerical Analysis $|$ Image Processing

\end{rSection} 

%----------------------------------------------------------------------------------------
%	TECHNICAL STRENGTHS SECTION
%----------------------------------------------------------------------------------------

\begin{rSection}{Skills and Interests}
\begin{myitemize}\setlength\itemsep{0.1em}
    \item \textbf{Language}: Python, C\texttt{++}, Java, JavaScript, Matlab and SQL;
    \item \textbf{Data analysis tools}: Scikit-learn, SciPy, Pandas, jMetal, Gephi;
    \item \textbf{DevOps tools}: Jenkins, Ansible, Travis-CI, AWS Elasticsearch, S3, Docker, Redis.
    % \item Interested in machine learning/data related positions as well as backend development and (automated) testing.
\end{myitemize}

% \begin{tabular}{ @{} >{\bfseries}l @{\hspace{6ex}} l }  
% Languages & \textbf{\textit{Proficient:}} Python $|$ Java $|$ \LaTeX; \textbf{\textit{Familiar with:}} JavaScript $|$ MatLab $|$ SQL $|$ C\texttt{++}\\
% Data Analytics & Scikit-learn, SciPy, Pandas, jMetal, Gephi, MPI\\
% DevOps Tools & Git, Jenkins, Ansible, Travis-CI, AWS Elasticsearch, S3, Docker, Redis \\
% \end{tabular}   

\end{rSection}

%-------------------------------------------------------------------------------
%	WORKING
%-------------------------------------------------------------------------------

\begin{rSection}{Intern and Research Experience}
\begin{rSubsection}{Spike: Predicting Breakdowns in Cloud Services}{Sep 2018- Apr 2019}
{Cooperation project with LexisNexis Legal \& Professional}{}
\item Fetched cloud services monitoring data in microservices systems;
\item Cleaned up the data and built machine learning models to predict the breakdowns 30 minutes ahead;
\item Got a alarm system for service spikes with recalls and precision of 75\% and above.
\end{rSubsection}


\begin{rSubsection}{Cross-Stan: Embedding Bayesian Modeling in Python and C++ Programs}{May 2018- Aug 2018}
{Internship at Facebook (Machine learning experience group)}{}
\item Created the tool {\it CxStan} to embed Bayesian Modeling engine, Stan, into Python and C++ programs;
\item Built a hierarchical framework to accelerate the  Monte Carlo Sampling for the Stan modeling;
\item Integrated the tool into the buck build tool, making the Stan engine as a service;
\item Applied the {\it CxStan} to some Facebook services' traffic policing.
\end{rSubsection}

\begin{rSubsection}{Automated Configurations for Cloud-based Workflows}{May 2017 - Aug 2017}
{North Carolina State University}{}
\item Presented a novel stochastic method for rapidly configuration cloud-based workflows;
\item Automatically deployed the workflow with more than 500 sub-tasks to AWS platform. Save up to 30\% economy cost within specific deadline requirement, compared to default greedy deployment policy in AWS.
% \vspace{0.5em}~\\
% \textbf{\it Skills: Software Architecture Analysis, Automated Configurations, Constraint Analysis}
\end{rSubsection}

\begin{rSubsection}{LACE Data Privatization Tools and its Application}{Aug 2016 - Nov 2016}{NSA funded project in \href{http://ai4se.net/index}{\textit{RAISE}} Lab}{}

\item Distributed a data anonymization  package in Python (see \url{http://tiny.cc/pydp}); tested package via Travis-CI;
\item Applied my package to education and medical data sets. Evaluate data set utility through supervised learning.
% \vspace{0.5em}~\\
% \textbf{\it Skills: Python, DevOps(Continuous Deployment), Data Deduplication, Data Cleaning}
\end{rSubsection}


\begin{rSubsection}{Fast Principal-component-analysis (F-PCA) Method for Flight Status Log} {May 2016 - Aug 2016}{Google Summer of Code program 2016}{}
\item Accepted by Google \href{https://developers.google.com/open-source/gsoc/}{\texttt{GSoC2016}}  program among 18,981 applicants \textbf{(accept rate: 6\%)};
% \item Basing on the \href{http://javapathfinder.sourceforge.net/}{\texttt{JPF}} platform, found the most promising sub-state space in the NASA \href{https://goo.gl/seVsNo}{\texttt{Brahms}} models.
\item Hierarchical clustering a dataset(flight status log) with more than 20M entries top-down and bottom-up. Create a PCA-like dimension reduction algorithm and speed it up by spark. Compared my own algorithm with PCA.
% \vspace{0.5em}~\\
% \textbf{\it Skills: Spark, Multi-objectives Optimization, Dimensionality Reduction, HAC}
\end{rSubsection}  

\newpage
%-------------------------------------------------- 
\begin{rSubsection}{Sampling vs. Searching in Search-based SE}{Dec 2014 - Aug 2017}
{North Carolina State University}{}
\item Created a sampling technique to solve the software product line problem. Deployed the algorithm into platform LSF by MPI; implemented a job schedule engine. Reduced the experiment time from 2 months CPU hrs into 11.5 hrs.
\item Modeled the Linux Kernel modules in CNF sets. Created the decision tree surrogate model. By combining SAT solvers, found a way to configure large software systems 2000x faster. Published the results in TSE.
\end{rSubsection}



\end{rSection}



%-------------------------------------------------------------------------------
%	PROJECTS
%-------------------------------------------------------------------------------

\begin{rSection}{Selected Coursework Projects}


%------------------------------------------------

% \begin{rSubsection}{Building Movie Recommendation System}{Aug 2015 - Dec 2015}{``Netflix Prize'' completion extension}{} 
\textbf{Building Movie Recommendation System:}
Built a movies recommendation system by training from 100 million
% \href{http://www.netflixprize.com/}{\texttt{Netflix ratings}} 
Netflix ratings
by Factorization Machine, SVM and ANN. Accelerated the learning process with HPC server.
% \item Crawled cast, critic reviews from rotten tomatoes and classified the movies basing on Jaro-Winkler Distance. Reduced the RMSE by up to 9\% with the help of external information.
% \vspace{0.5em}~\\
% \textbf{\it Skills: Scikit-learn, HPC, MPI, libFM}
% \end{rSubsection}

% %------------------------------------------------

% \begin{rSubsection}{Continuous Integration/Delivery Pipeline}{Aug 2015 - Dec 2015}{DevOps practice}{} 
\textbf{Continuous Integration/Delivery Pipeline:}
Basing on abstract syntax tree, created a regression test suite generator;
integrated Ansible scripts, Docker and Jenkins to build and deploy our ``sunrise-calculator'' app.
% \vspace{0.5em}~\\
% \textbf{\it Skills: Continuous Deploy, Ansible, Jenkins, Dockers, Test suite prioritization}
% \end{rSubsection}

\end{rSection} 

%--------------------------------------------------------------------------------------
%   Research Publications 
%--------------------------------------------------------------------------------------
\begin{rSection}{ Publications} 
\setenumerate[1]{label=[\arabic*]}
\begin{enumerate}[wide, labelwidth=!, labelindent=0pt]
\item \underline{Jianfeng Chen} and Tim Menzies. "On the Benefits of Restrained Mutation: Faster Generation of Smaller Test Suites" Submitted to IEEE/ACM International Conference on Automated Software Engineering (ASE 2019).
\item \underline{Jianfeng Chen}, Joymallya Chakraborty, Philip Clark, Kevin Haverlock, Snehit Cherian and Tim Menzies. "Predicting Breakdowns in Cloud Services (with SPIKE)". Submitted to Symposium on the Foundations of Software Engineering (ESEC/FSE 2019 - Industry Paper Track)
\item \underline{Jianfeng Chen}, and Tim Menzies. "RIOT: A Stochastic-Based Method for Workflow Scheduling in the Cloud." 2018 IEEE 11th International Conference on Cloud Computing (CLOUD). IEEE, 2018. (Accept rate: 15\%).
\item \underline{Jianfeng Chen}, Vivek Nair, Rahul Krishna, and Tim Menzies. "Sampling as a Baseline Optimizer for Search-based Software Engineering." IEEE Transactions on Software Engineering (2018).
\item \underline{Jianfeng Chen}, Vivek Nair, and Tim Menzies. "Beyond evolutionary algorithms for search-based software engineering." Information and Software Technology (2017).

\item Junjie Wang, Song Wang, \underline{Jianfeng Chen}, Tim Menzies, Qiang Cui, Miao Xie and  Qing Wang. "Characterizing Crowds to Better Optimize Worker Recommendation in Crowdsourced Testing.". IEEE Transactions on Software Engineering(2019).
\item Vivek Nair, Amrit Agrawal, \underline{Jianfeng Chen}, Wei Fu, George Mathew, Tim Menzies, Leandro Minku, Markus Wagner, and Zhe Yu. "Data-Driven Search-based Software Engineering." The Mining Software Repositories (MSR) 2018.
\item Tianpei Xia, Rahul Krishna, \underline{Jianfeng Chen}, George Mathew, Xipeng Shen, and Tim Menzies. "Hyperparameter optimization for effort estimation." Submitted Empirical Software Engineering (EMSE) 2018
\item Vivek Nair, Tim Menzies, and \underline{Jianfeng Chen}. "An (accidental) exploration of alternatives to evolutionary algorithms for SBSE." In International Symposium on Search Based Software Engineering, pp. 96-111. Springer, Cham, 2016.
\end{enumerate}
\end{rSection}

\end{document}
